\documentclass[]{tufte-handout}

% ams
\usepackage{amssymb,amsmath}

\usepackage{ifxetex,ifluatex}
\usepackage{fixltx2e} % provides \textsubscript
\ifnum 0\ifxetex 1\fi\ifluatex 1\fi=0 % if pdftex
  \usepackage[T1]{fontenc}
  \usepackage[utf8]{inputenc}
\else % if luatex or xelatex
  \makeatletter
  \@ifpackageloaded{fontspec}{}{\usepackage{fontspec}}
  \makeatother
  \defaultfontfeatures{Ligatures=TeX,Scale=MatchLowercase}
  \makeatletter
  \@ifpackageloaded{soul}{
     \renewcommand\allcapsspacing[1]{{\addfontfeature{LetterSpace=15}#1}}
     \renewcommand\smallcapsspacing[1]{{\addfontfeature{LetterSpace=10}#1}}
   }{}
  \makeatother
\fi

% graphix
\usepackage{graphicx}
\setkeys{Gin}{width=\linewidth,totalheight=\textheight,keepaspectratio}

% booktabs
\usepackage{booktabs}

% url
\usepackage{url}

% hyperref
\usepackage{hyperref}

% units.
\usepackage{units}


\setcounter{secnumdepth}{-1}

% citations

% pandoc syntax highlighting

% longtable

% multiplecol
\usepackage{multicol}

% strikeout
\usepackage[normalem]{ulem}

% morefloats
\usepackage{morefloats}


% tightlist macro required by pandoc >= 1.14
\providecommand{\tightlist}{%
  \setlength{\itemsep}{0pt}\setlength{\parskip}{0pt}}

% title / author / date
\title{Psychotherapy Study}
\author{Roberts Klotins and Thomas Gargot}
\date{02 April, 2016}


\begin{document}

\maketitle




\includegraphics{./cropped-banner_efpt.jpg} \tableofcontents

\section{EFPT}\label{efpt}

The European Federation of psychiatric trainees (EFPT) is an independent
federation of psychiatric trainees associations. It represents the
consensus of psychiatric trainees organizations across European
countries and advocates for what training should look like, regardless
of the country.

\href{http://efpt.eu/}{EFPT website} \footnote{\url{http://efpt.eu}}

\section{Presentation}\label{presentation}

The psychotherapy study run from 2013 to 2015 to understand the actual
opportunities and wishes for training in psychotherapy in Europe among
psychiatrist trainees in 19 countries.

\subsection{Presentations of the survey
:Past}\label{presentations-of-the-survey-past}

\begin{itemize}
\tightlist
\item
  ``The EFPT Psychotherapy working Group and its Survey on Psychotherapy
  Training 2013-2015'' was presented at the
  \href{http://www.rcpsych.ac.uk/pdf/CALC_MedPsych2015confbooklet.pdf}{Royal
  College of Psychiatrists} - faculty of Psychotherapy Annual Meeting
  22-24 April 2015 by Alina Petricean (UK, Romania)
\item
  ``The EFPT Psychotherapy working group'' was presented at the
  ``Portuguese Psychiatric National Congress in 2013 by Rita Silva
  (Portugal)
\item
  ``Should all psychiatrists be skilled to practice psychotherapy ?
  Thomas GARGOT (France) and Ekin Sonmez (Turkey),
  \href{http://efpt.eu/events/european-psychiatric-association-congress/}{European
  Psychiatric Association Congress}, March 12th - March 15th 2016,
  \href{http://efpt.eu/wp-content/uploads/2014/07/ES_TG_EPA_Psychotherapy.pdf}{slides}
\end{itemize}

\subsection{Projects}\label{projects}

\begin{itemize}
\tightlist
\item
  Submission Project : ``How psychiatrist trainees are trained in
  Psychotherapies in Europe ? A European trainee survey'', Thomas GARGOT
  (France), \href{http://eabct2016.org/}{European Association of
  Behavioural and Cognitive Therapies conference (EABCT)}, August 31st -
  September 3rd 2016, deadline 31st march
\item
  Submission Project : How psychiatrist trainees are trained in
  Psychotherapies in Europe ? A European trainee survey" in Italian,
  Diego Quattrone (Italy)
  \href{http://www.fiap.info/convegno2016/}{Italian Federation of
  Psychotherapy Associations (FIAP)}
\end{itemize}

\subsection{About data}\label{about-data}

You can download our data and our analysis script on
\href{https://github.com/r0bis/efptPWG}{GitHub}.\footnote{\url{https://github.com/r0bis/efptPWG}}
You can find ToDo list
\href{https://github.com/r0bis/efptPWG/blob/master/about-data/todolist.md}{here}.

\section{Actual recommandations}\label{actual-recommandations}

\subsection{EFPT Psychotherapy Working Group
statement}\label{efpt-psychotherapy-working-group-statement}

``A working knowledge of psychotherapy is an integral part of being a
psychiatrist and this must be reflected in training in psychiatry. All
trainees must gain the knowledge, skills and attitudes to be competent
in psychotherapy. Competence should be gained in at least one recognised
form of psychotherapy (of the trainees choice) and basic knowledge
should be gained in the other forms of psychotherapy to allow the
trainee to evaluate suitability for referral to specialist
psychotherapist. Training in psychotherapy must include supervision by
qualified therapists. A personal psychotherapeutic experience is seen as
a valuable component of training. It is crucial that trainees have
access to relevant psychotherapy experience to cater to the needs of the
appropriate patient group that the trainee is dealing with or is
expected to deal with in the future.

Relevant training authorities should ensure that time, resources and
funding are available to all trainees to meet the above mentioned
psychotherapy training needs."

Discussed and voted by EFPT delegates in Lisbon 1996, Tampere 1999,
Napoli 2001, Sinaia 2002, Paris 2003, Istanbul 2005, Gothenburg 2008,
Cambridge 2009 and Zurich 2013. See all EFPT statements\footnote{\url{http://efpt.eu/statements/}}.

\subsection{UEMS Psychiatry Board and Section Reports on
Training}\label{uems-psychiatry-board-and-section-reports-on-training}

Since its establishment the Board has worked on a number of important
aspects of psychiatric training in Europe. The work was normally carried
out through working groups which published their recommendations in the
form of reports. See all UEMS reports related to psychiatry
training\footnote{\url{http://uemspsychiatry.org/board/training-reports/}}.

\section{European results}\label{european-results}

\justify
\newthought{Aggregate European dataset is quite large.} Data are
collected in different tables, but because thay are of the same format
we can easily bind them together and perform summarisation and analysis
as needed. Further thoughts would be to see how countries differ along
the main parameters. For that we have to look carefully at the data from
Europe to formulate further questions of interest that can be answered
by this data set.

For example it would be very interesting to know if the 27 trainees
\footnote{Whole sample consisted of 571 trainees.} who have said that
they do not see Psychotherapy as an important part their professional
identity would still show interest in psychotherapy and would wish to
undertake training in psychoterapy if it were provided for free.

\subsection{Demographics}\label{demographics}

\newthought{Overall demographics seem to be as expected.} There are more
female trainees who tok part in the survey and this correlates with the
overall training trend. However the male trainees were definitely
represented in the sample.

\begin{marginfigure}
\includegraphics{psychotherapy_survey_analysis_PDF-tufte_files/figure-latex/Gender-1} \caption[Population by gender]{Population by gender}\label{fig:Gender}
\end{marginfigure}

\raggedright
Respondents came from 17 different countries. We still have not
collected data from some countries where response would be very
interesting. That is from the UK and Austria. in the \emph{UK} the
training system is well centralised and if we could distribute the
survey through the central address database we could ensure a large
dataset. \emph{Austria} to our knowledge is the only country where all
psychiatrists must train in psychohterapy too.

\begin{figure}
\includegraphics{psychotherapy_survey_analysis_PDF-tufte_files/figure-latex/country_Response-1} \caption[Response in countries where survey was conducted]{Response in countries where survey was conducted.}\label{fig:country_Response}
\end{figure}

\marginnote[- 35mm]{Number of answers:  563 from  18  countries  Mean number of answers per country: 31 Top five response counts from: Romania, France, Slovenia, Czech Republic and Italy}

We did not forget to check age distribution of our respondents.

\begin{figure}
\includegraphics{psychotherapy_survey_analysis_PDF-tufte_files/figure-latex/age_Distribution-1} \caption[Age distribution]{Age distribution}\label{fig:age_Distribution}
\end{figure}

\newthought {By age the respondents} showed maximal response rate
between ages 26 and 32. However this was expected as it reflects on
general population of psychiatry trainees. Median age however was 31 and
this shows the significant number of older trainees - a few well into
their fifties who voiced their opinion regarding the psychotherapy
issue. It would be interesting to establish correlations among odler and
younger trainee populations and perhaps do it on country by country
basis too. We could try to find a cutoff point at which core opinions
diverge. Or establish that there is no age difference.

\newthought{Year in training} generally reflected the age distrubution
as well. We also collected data from among trainees who had spent more
than eight years in training and from those who had recently finished
their training.\footnote{RF in the diagram means \emph{Recently
  Finished}.} Recent in this case was defined as within the last 5
years.

\begin{figure}
\includegraphics{psychotherapy_survey_analysis_PDF-tufte_files/figure-latex/YOT-1} \caption[Population by Year in training]{Population by Year in training}\label{fig:YOT}
\end{figure}

\newthought{Was the trainee in therapy?} A further interesting
characteristic in our sample was to identify whether the trainees were
receiving personal therapy or not. As expected most of the respondents
(59.02\%) had no experience of being in personal therapy. Whether the 40
\% who had had some experience in therapy were an unusually large
proportion for the population of psychiatric trainees is not known yet.

\begin{marginfigure}
\includegraphics{psychotherapy_survey_analysis_PDF-tufte_files/figure-latex/inTherapy-1} \caption[Proportion of trainees in therapy]{Proportion of trainees in therapy}\label{fig:inTherapy}
\end{marginfigure}

\newthought{Was the trainee in psychotherapy training?} This question
was somewhat multifacted. We recognise that some trainees might have
partial ir full training in psychotherapy training within their training
programme in psychiatry. However by experience we also knew that many
trainees undertook training outside their training programme - most
often paying for that out of their own pocket. We therefore asked to
tell who were in psychotherapy training, whether their training had been
completed previously. To account for unusual situations we also provided
an opportuinty to choose \emph{Other} in response and to provide a
textual summary of that.

\newthought{And as we can see} this can be quite interesting. Lorem
ipsum dolor sit amet, sed ullamcorper imperdiet ac nullam ut, egestas,
vestibulum vitae ex sem senectus mauris. Viverra egestas lorem fames,
felis netus. Leo mauris hac sed fermentum quis justo. In in sapien non
et ut ridiculus sed mi velit sapien. Interdum dictum in blandit. At
pellentesque egestas molestie auctor, ante nibh nec elit nec ultrices
sem. Taciti non ligula, nascetur adipiscing conubia tellus rutrum sit
sociis. Erat senectus dui dictumst pellentesque et orci aenean tincidunt
sapien mauris dis, scelerisque convallis.

\begin{figure}
\includegraphics{psychotherapy_survey_analysis_PDF-tufte_files/figure-latex/in_Training-1} \caption[Whether respondents were in training]{Whether respondents were in training}\label{fig:in_Training}
\end{figure}

\newpage

\begin{table}[!t]
\centering
\begin{tabular}{lp{4in}}
  \toprule
 & Other answers \\ 
  \midrule
1 & no \\ 
  2 & I have done some training in psychotherapy durimg my training programe in psychiatry \\ 
  3 & In our training programme in psychiatry we have the first 2 years initiative sessions in the different psychotherapy options. \\ 
  4 & During our training programme in psychiatry, the first 2 years we get initiation sessions on the different types of psychotherapy \\ 
  5 & I will undertake training in psychotherapy, probably next year on my own initiative \\ 
  6 & training interrupted and finished \\ 
  7 & completed training in the past, on my own initiative \\ 
  8 & I have completed a three-year training in psychotherapy within my training programme in psychiatry, but I am not a qualified psychotherapist. \\ 
  9 & I have applied for training \\ 
  10 & I will be undertaking training \\ 
  11 & I will next year \\ 
  12 & mindfulness-based stress reduction programme \\ 
  13 & Around 50 hours overall discussing cases from the psychodynamic point of view \\ 
  14 & I undertook training in the past \\ 
  15 & Psychoanslysis \\ 
  16 & starting next year \\ 
  17 & nothing \\ 
  18 & We have some training in psychodynamic therapy, CBT, group and family training but we are not trained to be able to work as psychotherapists \\ 
  19 & I've started training in Cbt on my own initiative but had to stop from personal reasons \\ 
  20 & i have undertaken training in psychotherapy on my own iniative, but I have since interrupted \\ 
  21 & I had one year of psychodynamic training in uk. I've returned to Romania and I am not undertraining any training here. \\ 
  22 & i am undertaking training in psychotherapy after comleted specialisation of psychiatry \\ 
  23 & I COMPLETED THE INTRO COURSE IN GROUP PSYCHOTHERAPY \\ 
  24 & just magistral classes, no assessments of work with patients \\ 
   \bottomrule
\end{tabular}
\caption{Other responses to the question whether trainees have trained in psychotherapy} 
\end{table}

This was an overview of the 24 answers provided in the other option.

\section{What psychotherapy modality have you trained
in?}\label{what-psychotherapy-modality-have-you-trained-in}

This was an optional question presented to only those 547 respondents
who had answered positively to the previous question i.e.~those who had
trained in any modality of psychotherapy. We wanted to get a qualitative
overview as to what therapies they had trained in. Since this number is
quite large we can only present you with a sample summary of these 329
answers, but to give a feel for what they were like we have included the
first 4 rows of these answers.

\begin{table}[!h]
\centering
\begin{tabular}{lp{3.5in}}
  \toprule
 & Experience in modality \\ 
  \midrule
1 & psychoanalysis \\ 
  2 & Cognitieve Behavioural Therapy \\ 
  3 & We received a introduction course in cliented centered en systemic therapy yet. \\ 
  4 & Family and Marital Therapy \\ 
   \bottomrule
\end{tabular}
\caption{What modality have trainees had experience with} 
\end{table}

\subsection{Summary of psychotherapeutic methods that trainees had
experience
in}\label{summary-of-psychotherapeutic-methods-that-trainees-had-experience-in}

We summarised data along the following categories after reading them
manually. Lorem ipsum dolor sit amet, sed non ut pharetra turpis
nascetur volutpat velit. Nec suspendisse sollicitudin. Vulputate commodo
eu dignissim donec id. Eu fames sit non neque mi hac, mauris ipsum. In
non praesent netus ac per velit morbi. Diam imperdiet velit. Turpis ut,
egestas massa purus eros magnis proin. Imperdiet hac efficitur
pellentesque consequat, non sed. Pellentesque posuere aliquet eu vel
eget eros proin, pellentesque. Porttitor in libero ut tempus ac purus mi
congue lectus turpis. Pharetra ultrices nec cras proin pharetra risus
eget risus nulla arcu consequat. Odio suspendisse, proin pharetra taciti
eros augue tempor eu. Massa, suspendisse est nunc felis habitant,
imperdiet ac mus in sed donec, maximus velit. Faucibus sit magna blandit
nibh. Massa mauris vestibulum netus praesent aliquam., Eu vestibulum ut
nibh odio nullam at, consectetur maximus. Eu id congue, duis sed a
facilisis interdum in eget. A, et cras ex ultrices eu dictum netus odio?
Mi aliquam. Turpis suscipit metus himenaeos sodales. Magna orci,
consequat malesuada, ut et dolor adipiscing at at? Magnis orci, nunc
amet venenatis faucibus elit. Semper amet mi quisque curabitur tempus
eu, convallis efficitur. Porta, nunc dictumst tincidunt, laoreet et,
fringilla at faucibus et non sagittis magna. Varius, neque blandit ac
sed in et. Eleifend nisl sed at elit quisque per at. Ante tortor id
dictum litora. Varius aliquet consectetur morbi, hendrerit ut. Nec
himenaeos vivamus sed proin ultrices leo finibus egestas justo lectus
donec. Sed et et tempor duis.

\section{Your preferences - the second group of
questions}\label{your-preferences---the-second-group-of-questions}

\newthought{Is psychotherapy training important for your professional identity as 
a psychiatrist?} was the key question in the mind of several creators of
this survey. We expected quite clear preference one way or the other,
but we wanted to also give an opportunity to not have to choose. We
expected that there might be quite a large group of people who would
answer I don't know. However we were somewhat surprised by the results:

\begin{figure}
\includegraphics{psychotherapy_survey_analysis_PDF-tufte_files/figure-latex/prof_Identity-1} \caption[Importance of psychotherapy for professional identity]{Importance of psychotherapy for professional identity}\label{fig:prof_Identity}
\end{figure}

\section{Concluding notes}\label{concluding-notes}

\justify
[1] ``Lorem ipsum dolor sit amet, turpis tempor, habitant sed placerat
semper dolor parturient. Nisl etiam bibendum nisl sagittis velit
quisque, neque. Mauris vivamus eu, ipsum, etiam, in elementum, id a.
Lorem nulla suscipit ac ac enim dictumst, risus. Sed et, nibh enim diam
in est, mauris et varius praesent, tellus. Pretium hendrerit tincidunt
eros tristique justo sed aenean, condimentum libero nunc quis.
Ullamcorper a metus ut iaculis nec blandit molestie, ut. Elit quis sed
nec suscipit vestibulum augue nam nulla per.''

\newpage

\subsection{Research}\label{research}

UEMS recommendations states that research methodology should be included
in psychotherapy training.

\subsection{Bibliography}\label{bibliography}

\begin{itemize}
\tightlist
\item
  You are welcome to join our
  \href{https://www.zotero.org/groups/480046}{zotero group}\footnote{\url{https://www.zotero.org/groups/480046}}
\end{itemize}



\end{document}
